% !TEX root =  master.tex
\chapter{Grundlagen}
\section{Arc Raiders \& Gaming Tools}

\subsection{Companion Applications im digitalen Ökosystem}

Companion Applications haben sich als eigenständige Software-Kategorie etabliert, die primäre Anwendungen durch zusätzliche Funktionalität ergänzt, ohne das Hauptprodukt zu ersetzen. Im Gaming-Kontext erfüllen diese Anwendungen mehrere Schlüsselfunktionen: Sie erweitern die Spielerfahrung über die Session hinaus, bieten Planungs- und Analysewerkzeuge und ermöglichen Social Features außerhalb der Spielumgebung.

Companion Apps für Gaming-Konsolen verzeichneten seit 2020 ein signifikantes Wachstum, wobei die PC-Gaming-Plattform Steam mit etwa 33,4 Millionen Downloads im zuletzt gemessenen Quartal führend war~\cite{Statista2024ConsoleApps}. Diese Entwicklung zeigt, dass Spieler bereit sind, zusätzliche Tools zu nutzen, um ihre Gaming-Erfahrung zu optimieren.

Die technische Architektur von Companion Apps variiert je nach Anwendungsfall:
\begin{itemize}
    \item \textbf{Offizielle Plattform-Apps:} Direkte Integration mit Herstellersystemen (PlayStation App, Xbox App, Steam Mobile)
    \item \textbf{Game-spezifische Tools:} Fokus auf ein einzelnes Spiel oder Franchise
    \item \textbf{Community-getriebene Lösungen:} Von Spielern entwickelte Third-Party-Tools
\end{itemize}

\subsection{Gaming Companion Apps: Funktionale Kategorisierung}

Basierend auf der Analyse existierender Lösungen lassen sich Gaming Companion Apps in folgende funktionale Kategorien einteilen:

\textbf{Informations- und Datenbank-Tools:} Diese Tools bieten Zugriff auf Spieldaten wie Item-Statistiken, Charakterwerte oder Mechaniken. Apps wie Handbook for EFT bieten Spielern offline verfügbare Informationen zu Karten, Munitionsvergleichen, Waffen-Performance, Ausrüstung, Quest-Guides und Key-Informationen~\cite{HandbookEFT2024}.

\textbf{Progress-Tracking-Systeme:} Apps wie The Hideout: Tarkov Sidekick ermöglichen Quest-Tracking, Hideout-Modul-Verwaltung und Team-Quest-Tracking, wo Spieler den Status ihrer Freunde sehen können~\cite{TheHideoutApp2024}.

\textbf{Markt- und Wirtschafts-Tools:} Funktionen wie Flea Market-Preisanzeige, Preishistorien bis zu einem Jahr zurück und Preis-Alerts für Flea Market-Preise helfen Spielern bei wirtschaftlichen Entscheidungen~\cite{TheHideoutApp2024}.

\textbf{Karten- und Navigationshilfen:} Interactive Maps-Apps bieten detaillierte, community-erstellte Karten mit Loot-Spots, Extraktionspunkten, Key-Locations und Quest-Details~\cite{GameMapsOverwolf2024}.

\textbf{Build-Planer und Kalkulatoren:} Weapon Builder und Damage Calculator ("`Tarkov'd Simulator"') ermöglichen theoretisches Durchspielen von Szenarien vor der Implementierung im Spiel~\cite{TheHideoutApp2024}.

\subsection{Extraction Shooter: Genre-spezifische Anforderungen}

Extraction Shooter basieren auf Konzepten der Verlustaversion und verzögerten Gratifikation, wobei jeder Raid ein Risiko darstellt und der erfolgreiche Abschluss eines wertvollen Durchgangs intensive dopamingesteuerte Befriedigung freisetzt~\cite{InsiderGaming2025ExtractionShooters}. Diese psychologischen Mechanismen erzeugen spezifische Anforderungen an unterstützende Tools.

\textbf{Risiko-Management:} Die Permadeath-Mechanik von Extraction Shootern erfordert sorgfältige Planung. Companion Apps können helfen, Risiken zu minimieren durch:
\begin{itemize}
    \item Vorausschauende Ressourcenplanung
    \item Optimale Route-Planung zur Minimierung von Begegnungen
    \item Wert-Kalkulation von Loot vs. Risiko
\end{itemize}

\textbf{Komplexitätsreduktion:} Das Extraction-Shooter-Genre gilt aufgrund seiner Komplexität als schwer zugänglich für den Massenmarkt~\cite{IconEra2024ExtractionMarket}. Arc Raiders hat durch seine Betonung von Teamsynergien und intuitiveren Spielerführungssystemen einen zugänglicheren Einstiegspunkt geschaffen~\cite{InsiderGaming2025ExtractionShooters}.

Der aktuelle Markt konzentriert sich auf Escape from Tarkov (ca.~60.000 gleichzeitige Nutzer), Hunt Showdown (ca.~20.000) und Dark and Darker (ca.~10.000), zusammen etwa 100.000 gleichzeitige Nutzer~\cite{IconEra2024ExtractionMarket}. Die Herausforderung für neue Titel liegt in der Etablierung eigener Tool-Ökosysteme.

\subsection{Arc Raiders: Technische Charakteristika und Systeme}

ARC Raiders ist ein 2025 veröffentlichter Third-Person-Extraction-Shooter, entwickelt mit der Unreal Engine 5 für PlayStation 5, Windows und Xbox Series X/S~\cite{WikipediaARCRaiders2025}. Bis zum 11.~November 2025 hatte das Spiel weltweit über vier Millionen Exemplare verkauft~\cite{WikipediaARCRaiders2025}, was eine signifikante Nutzerbasis für Companion-Tools darstellt.

Das Spiel implementiert mehrere Systeme, die durch externe Tools unterstützt werden können:

\textbf{Quest-System:} Spieler erfüllen Quests für Händler mit unterschiedlichen Motiven und Agenden, was sich in komplexen Quest-Chains mit Abhängigkeiten manifestiert~\cite{SteamARCRaiders2025}. Das Ingame-Interface zeigt nur aktive Quests, was einen Gesamtüberblick erschwert.

\textbf{Crafting und Ressourcen:} Spieler müssen Workshop-Stationen upgraden und Blueprints lernen, um fortgeschrittenere Items zu craften~\cite{SteamARCRaiders2025}. Bei begrenztem Inventarplatz ist strategisches Ressourcen-Management essentiell.

\textbf{Skill-Progression:} Der ARC Raiders Skill-Tree verzweigt sich in drei Pfade: Survival, Mobility und Conditioning~\cite{SteamARCRaiders2025}, was Langzeitplanung erfordert.

\textbf{Multiplayer-Koordination:} Das Spiel unterstützt nahtloses Cross-Platform-Spiel zwischen PlayStation, Xbox und PC, wobei Spieler in Squads bis zu drei Personen oder solo spielen können~\cite{SteamARCRaiders2025}.

\subsection{Referenzimplementierungen: Tarkov Companion Ecosystem}

Das Escape from Tarkov Ökosystem bietet wertvolle Erkenntnisse für die Entwicklung von Companion Apps.

Tarkov Companion als Overwolf-App bietet Quest-Management sortiert nach Händler, Location oder Status, Browse-Funktionen für Karten mit Extraktionen, Loot-Hotspots und Quest-Items sowie Key-Suche und Hideout-Upgrade-Tracking~\cite{OverwolfTarkovCompanion2024}. Diese Features repräsentieren den aktuellen Standard für Extraction-Shooter-Companion-Apps.

Tarkov.dev bietet ein freies, community-erstelltes und Open-Source-Ökosystem von Escape from Tarkov-Tools inklusive Informationen zu Items, Crafts, Barters, Maps, Loot-Tiers, Hideout-Profiten und einer freien API~\cite{TarkovDev2024}. Die Verfügbarkeit einer API ermöglicht die Entwicklung vielfältiger Third-Party-Tools.


\section{Moderne Web-Technologien}
  \subsection{Typescript}
  kurz halten

  Type Safety in großen Projekten

  Developer Experience

  \subsection{React}
  Komponentenbasiertes UI

  Komposition over Inheritance

  \subsection{Full Stack React Frameworks}
  Nextjs als React "Backend" Framework, 

  SSR vs. SSG vs. ISR vs. PPR (Partial Prerendering)

  file based routing

\section{UI/UX Frameworks \& Design System}

  \subsection{Tailwindcss}
  Utility First CSS Framework, inline

  v4 mit "Just in Time" Compiler und global.css

  \subsection{Moderne Komponenten-Bibliotheken}
  Shadcn/using

  unterschied zu "klassischen" Bibliotheken wie MUI bootstrap

  \subsection{React-Flow}
  Visualisierung von Graphen/Netzwerken

\section{State Management \& Data Fetching}

  \subsection{State Management/State Stores}
  Bibliotheken wie zustand

  client side
  
  \subsection{react-query / tanstack-query als Daten-Fetching Library}
  Server State Management

  Caching Strategien

  Optimistic Updates

\section{Datenbank und Backend Design}

  \subsection{Supabase}
  PostgreSQL-basiert

  Real-time Capabilities

  Row Level Security

  edge functions (Serverless Functions)

  \subsection{Orm - Drizzle}
  ORM's

  Type-Safety

  code-first approach

\section{Deployment \& DevOps}

  \subsection{Git basierter Workflow}

  \subsection{Vercel}
  Deployment von Nextjs Applikationen

  weitere features

  CI/CD Pipelines

\section{Testing Frameworks \& Strategien}

  \subsection{Vitest}
  Unit und Integration Tests

  fast, modern, built for TS

  \subsection{Cypress}
  End-to-End Testing

  real browser testing
