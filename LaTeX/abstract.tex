% !TEX root =  master.tex
%%
%% @stud: sprachspezifische Anpassung
%%
\chapter*{Abstract}
\addcontentsline{toc}{chapter}{Abstract}

The gaming market, projected to grow to 600.74 billion USD by 2030, is driving 
increasing demand for companion applications that address the complexity of 
modern games. Extraction shooters such as Arc Raiders confront players with 
multifaceted progression systems, limited inventory capacities, and squad-based 
gameplay, yet lack official planning tools to support these challenges.

The central research question of this thesis is: How can a web-based companion 
app be systematically developed to reduce information asymmetry in quest lines 
and enable efficient resource management in extraction shooters? Methodologically, 
an agile four-phase model was applied: Requirements Engineering utilizing user 
stories following INVEST criteria, Architecture and Design, Implementation with 
the T3 Stack (React/Next.js, TypeScript), and Evaluation through 
\ac{E2E} testing combined with \ac{SUS}-based user surveys.

The developed ArcDéx Companion App achieved a fulfillment rate of 96.2\% for 
the defined acceptance criteria. Core contributions include: (1) a quest 
management system featuring a Kanban board and flow chart visualization for 
dependency graphs (UX score 4.42/5), (2) a material calculator for aggregated 
resource computation across quests and workstation upgrades (UX score 4.33/5), 
(3) a recycling calculator with reverse engineering functionality and graph 
visualization (UX score 4.42/5). The qualitative evaluation using the \ac{SUS} 
yielded a score of 73.75 points, corresponding to the \glqq Good\grqq{} category.

This thesis contributes to research in the field of gaming companion applications 
by demonstrating transferable architectural patterns for complex progression 
systems. The combination of graph-based dependency visualization and calculator 
tools addresses challenges prevalent in many modern games featuring meta-progression 
mechanics. The identified limitation of manual data entry opens potential for 
extension through native platforms such as Overwolf with automatic game event 
capture capabilities.
\newpage
\chapter*{Zusammenfassung}
\addcontentsline{toc}{chapter}{Zusammenfassung}

Der Gaming-Markt mit einem prognostizierten Wachstum auf 600,74 Milliarden USD 
bis 2030 treibt die Nachfrage nach Companion Apps für komplexe Spiele stetig 
voran. Extraction Shooter wie Arc Raiders konfrontieren Spieler mit vielschichtigen 
Progressionssystemen, begrenzten Inventarkapazitäten und squadbasiertem Gameplay, 
für das offizielle Planungswerkzeuge fehlen.

Die zentrale Forschungsfrage dieser Arbeit lautet: Wie kann eine webbasierte 
Companion App systematisch entwickelt werden, die Informationsasymmetrie bei 
Quest-Lines reduziert und effizientes Ressourcenmanagement in Extraction Shootern 
ermöglicht? Methodisch wurde ein agiles Vier-Phasen-Modell angewandt: 
Requirements Engineering mit User Stories nach INVEST-Kriterien, Architektur 
und Design, Implementierung mit dem T3 Stack (React/Next.js, TypeScript) 
sowie Evaluation durch \ac{E2E}-Tests und \ac{SUS}-basierte Nutzerbefragung.

Die entwickelte ArcDéx Companion App erreichte eine Erfüllungsquote von 96,2\% 
bei den definierten Akzeptanzkriterien. Kernleistungen umfassen: (1) 
Quest-Management mit Kanban-Board und Flow-Chart-Visualisierung für 
Abhängigkeitsgraphen (UX-Score 4,42/5), (2) Material-Calculator für aggregierte 
Ressourcenberechnung über Quests und Workstation-Upgrades (UX-Score 4,33/5), 
(3) Recycling-Kalkulator mit Reverse-Engineering-Funktionalität und 
Graph-Visualisierung (UX-Score 4,42/5). Die qualitative Evaluation mit dem 
\ac{SUS} ergab einen Score von 73,75 Punkten, was der Kategorie \glqq Good\grqq{} 
entspricht.

Die Arbeit trägt zur Forschung im Bereich Gaming Companion Apps bei, indem sie 
übertragbare Architekturmuster für komplexe Progressionssysteme demonstriert. 
Die Kombination aus Graph-basierter Abhängigkeitsvisualisierung und 
kalkulatorischen Werkzeugen adressiert Herausforderungen, die in vielen 
modernen Spielen mit Meta-Progression auftreten. Die identifizierte Limitation 
der manuellen Dateneingabe eröffnet Erweiterungspotenzial durch native 
Plattformen wie Overwolf mit automatischer Spielereigniserfassung.
